\documentclass[10pt,letterpaper]{article}
\usepackage[latin1]{inputenc}
\usepackage{amsmath}
\usepackage{amsfonts}
\usepackage{amssymb}
\usepackage{graphicx}
\title{PHYS-1120 Chapter 27 \\ Faraday's Law}
\author{Dawson Beatty}
\begin{document}
	\maketitle
	Faraday showed that there's another way to make a magnetic field: A \textit{changing} electric field makes a magnetic field.
	
	\subsection*{EMF $\varepsilon$}
	Defined as a voltage difference, $\Delta V = Ed$.  Think of it as a battery voltage. 
	
	$$
	\boxed{\varepsilon = \oint_{\mathcal{L}} \vec{E} \cdot \text{d} \vec{\mathit{l}}}
	$$
	
	\subsection*{Magnetic Flux $\Phi_\text{B}$}
	
	$$
	\Phi_\text{B} = \int_\text{S} \vec{B} \cdot \text{d}\vec{A} = \vec{B} \cdot \vec{A} = BA \cos \theta
	$$
	
	The right two terms above are \textit{only} true if $B$ is constant, and $A$ is flat. 
	
	Units: $[\Phi] = I\cdot m^2 = $ weber (Wb)
	
	\subsection{Faraday's Law}
	An induced emf $(\varepsilon)$ is created by changing magnetic flux
	
	$$
	\boxed{\varepsilon_\text{N loops} = - N\frac{\text{d} \Phi_\text{M}}{\text{dt}}}
	$$
	
	B = constant $\Rightarrow \varepsilon = 0$
	B is changing with time $\Rightarrow |\varepsilon| = \left|\frac{\text{d}\Phi}{\text{dt}}\right|$
	
	\subsubsection*{Changing the Magnetic Flux}
	\begin{itemize}
		\item Change $B$ (increase or decrease magnitude of B-field)
		\item Change $A$ (altering the shape of the loop)
		\item Change the angle $\theta$ between $B$ and the vector $A$, (by rotating the loop)
	\end{itemize}
	
	\subsection*{Lenz's Law}
	States that the induced emf induces a current that flows in the direction which creates an induced B-field that \textit{opposed the change} in flux. Lenz's law doesn't like change, wants to keep the status quo. 
	
	\subsection*{Eddy Currents}
	If a piece of metal and a B-field are in relative motion such that $\Phi$ changes through some loop within the metal, the changing $\Phi$ creates an emf which drives a current $I$. This is called an \textit{Eddy current}.  This current and the B-field will always cause a magnetic force to slow the motion of the metal. 
	$$\vec{F} = I\vec{L} \times \vec{B}$$
	
	
	
\end{document}