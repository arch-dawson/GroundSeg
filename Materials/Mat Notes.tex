\documentclass[10pt,letterpaper]{article}
\usepackage[latin1]{inputenc}
\usepackage{amsmath}
\usepackage{amsfonts}
\usepackage{amssymb}
\usepackage{graphicx}
\usepackage{pgfplots}
\usepackage{soul}
\author{Dawson Beatty}
\title{Materials Notes \\ ASEN 1022}
\setcounter{section}{1}
\begin{document}
	\maketitle
	\section{Atomic Bonding}
	\subsubsection*{Frequency of Isotopes}
	\begin{align*}
		& n_{\text{total}} = an_1 + bn_2 & 1 = a + b
	\end{align*}
	Solve for $a$ or $b$ in the right equation, sub into the left equation. 
	\subsection*{Forces}
	\subsubsection*{Coulombic Attraction}
	$$
	F_c = \frac{-K}{a^2}
	$$
	where $K = k_0(Z_1 q)(Z_2 q)$
	\subsubsection*{Repulsive Force}
	$$
	F_R = \lambda e^{\frac{-a}{r}}
	$$
	
	The net bonding force is $F = F_c + F_R$. The equilibrium bond length $a_0$ occurs at the point where the forces balance.
	\subsubsection*{Bonding Energy}
	Bonding energy $E$ is related to the bonding force through the differential expression 
	$$
	F = \frac{dE}{da}
	$$
	The equilibrium bond length occurs at a minimum in the energy curve, where the force equals zero. 
	$$
	F = 0 = \bigg(\frac{dE}{da} \bigg)_{a=a_0}
	$$
	It follows that $a_0$ is the sum of two atomic radii: 
	$$
	a_0 = r_{Na^+}+r_{Cl^-}
	$$
	\subsubsection*{Coordination Number}
	Coordination number determines the number of atoms that can be packed around a central atom. 
	
	Radius ratio is calculated as $\frac{r}{R}$, where $r$ is the radius of the smaller ion, and $R$ is the radius of the larger. 
	\subsubsection*{Covalent Bonds}
	To determine the reaction energy, take the final bond energy from the new bonds formed, and subtract the initial energy of the bonds being broken. 
	
	\textit{Example: } Length of a polyethylene molecule, \st{(}$C_2H_4$\st{)}$_n$, where $n = 500$. 
	
	Bond length $l = (C -- C \: \text{bond length}) \times \sin(\frac{109.5}{2}^{\circ})$
	
	Total length $L = 500 \times 2 \times l$
	\subsubsection*{Bonding Energy Curve}
	Common way to describe the curve is 
	$$
	E = -\frac{K_A}{a^6} + \frac{K_R}{a^{12}}
	$$
	Take the derivative and set equal to zero, solve for $a$ to find the equilibrium bond length. 
	
	\section{Perfect Crystalline Structures}
	Seven Crystal Systems, can be stacked together in the 14 \textit{Bravais Lattices}.
	\begin{tabular}{|l|c|c|}
		\hline 
		\multicolumn{3}{|l|}{Edge Length($a$), Atomic Radius($r$) and atoms/unit} \\ 
		\hline 
		\textbf{Crystal Structure} & \textbf{a vs. r} & \textbf{Atoms/Unit cell}\\ 
		\hline 
		bcc & $a=\frac{4r}{\sqrt{3}}$ & 2 atoms/unit\\ 
		\hline 
		fcc & $a=\frac{4r}{\sqrt{2}}$  & 4 atoms/unit\\ 
		\hline 
		hcp & $a=2r$ & 2 atoms/unit\\ 
		\hline 
	\end{tabular} 
	
	\subsubsection*{Triangles in $\mathbb{R}^3$}
	For a triangle with vector sides $u$ and $v$ originating at one vertex, the area can be written as 
	$$
	A = \frac{1}{2} | u \times v |
	$$
	\subsubsection*{Diffraction}
	$$
	n \lambda = 2d \sin \theta
	$$
	Where $d$ is spacing between adjacent crystal planes, $\theta$ is the angle of scattering, $\lambda$ is the radiation wavelength, $n \: \epsilon \: \mathbb{Z}$.
	\subsubsection*{Interplanar Spacing}
	\textit{Cubic:}
	$$
	d_{hkl} = \frac{a}{\sqrt{h^2 + k^2 + l^2}}
	$$
	Where $a$ is the side length of a unit cell, and $h,k,l$ come from miller indices of the plane. 
	
	\textit{Hexagonal: }
	$$
	d_{hkl} = \frac{a}{\sqrt{\frac{4}{3}(h^2 + hk + k^2)+l^2(\frac{a^2}{c^2})}}
	$$ 
	Where $a$ and $c$ are the lattice parameters. 
	
	
	\section{Crystal Defects and Noncrystalline Structure-Imperfection}
	\subsection{Solid Solutions}
	Can think of metal alloys as solid solutions.  Called a \textit{substitutional solid solution} because some atoms are swapped for others in the lattice sites. Only occurs when these rules are met: 
	\subsubsection{Hume-Rothery Rules}
	\begin{enumerate}
		\item Less than about 15\% difference in atomic radii.
		\item The same crystal structure
		\item Similar electronegativities 
		\item The same valence
	\end{enumerate}
	$$
	\% \text{Difference} = \frac{R - r}{R} \times 100
	$$
	
	If one or more rule is violated, only partial solubility is possible. Solutions can either be \textit{random solid solutions}, (differing particles randomly mixed in) or \textit{ordered solid solutions}. (Where different particles take the face positions of an \textit{fcc} lattice, for example.)
	
	\subsubsection{Interstitial Solid Solutions}
	Interstitial solid solutions occur when one atom is way smaller than the others, and can fit into the gaps in the lattice structure. 
	
	Ideally an interstitial atom would just be touching the other atoms in the structure. This ideal radius can be calculated: 
	$$
	r_{interstitial} = \frac{1}{2} a - R
	$$
	Where $a$ is the length of the unit-cell edge, and $R$ is the radius of a larger atom. 
	
	Net charge of the substance must remain the same: to add 2 $Al^{3+}$ ions, 3 $Mg^{2+}$ ions must be removed. This leads to points in the structure with no ions present. 
	
	\subsubsection{Non-Stoichiometric Compounds}
	Iron Oxide, $Fe_{1-x}O$ with $x \simeq .05$ is an example of a \textit{nonstoichiometric compound}, since it has both $Fe^{2+}$ and $Fe^{3+}$ ions occupying the cation sites. Ideally there would be an equal amount of $Fe^{2+}$ and $O^{2-}$. 
	\subsection{Point Defects}
	\subsubsection{Schottky Defect}
	\begin{itemize}
		\item A pair of oppositely charged ion vacancies. 
		\item Ensures local charge neutrality
	\end{itemize}
	\subsubsection{Frenkel Defect}
	\begin{itemize}
		\item Vacancy-interstitialcy combination
		\item One slot with a vacancy, one slow with two ions crammed in.
	\end{itemize}
	\subsection{Linear Defects}
	Also known as \textit{dislocations}, given the symbol $\bot$. $\bot$ represents an extra half-plane of atoms.
	\subsubsection{Burgers Vector}
	The Burgers Vector $b$ is the displacement vector to close a stepwise loop around the defect. In a perfect crystal, the loop stop and start at the same place. The vector represents the magnitude of the defect. 
	
	The magnitude of $b$ in simple metal structures is the repeat distance along the highest atomic density direction. 
	\subsubsection{Edge Dislocation}
	Dislocation line runs along the edge of an extra row of atoms.  In this case the Burgers vector is perpendicular to the dislocation line. 
	\subsubsection{Screw Dislocation}
	Comes from the spiral stacking of crystal planes around the dislocation line.  Burgers vector is parallel to the dislocation line. 
	
	The two above are the perfect extremes, many are mixed variants of the two. Burgers vector will neither be parallel or perpendicular to the dislocation line. 
	\subsection{Planar Defects}
	\subsubsection{Twin Boundary}
	Separates two crystalline regions that are structurally mirror images. 
	\subsubsection{Hirth-Pound Model}
	Elaborate ledge system instead of smooth planes.  
	
	\textit{Grain Boundaries} are the most important here, which is the region between two adjacent single crystals or grains. 
	\subsection{Tilt Boundary}
	Occurs when two adjacent grains are titled a few degrees relative to each other.  Includes a few isolated edge vacancies.  Regions will consist of regions of good correspondence, separated by \textit{grain-boundary dislocations}, a type of linear defect. 
	
	Distance between isolated edge dislocations is given by $D = \frac{b}{\theta}$, where b is the length of the Burgers vector, and $\theta$ is the tilt between adjacent crystal sheets. 
	
	It's useful to have an index of grain size, typically use \textit{grain-size number}, $G$.
	$$
	N = 2^{G-1}
	$$
	Where $N$ is the number of grains observed in an area of 1 in.$^2$ (645 mm$^2$) on a photomicrograph taken at $100 \times$ magnification. 
	
	Can also count the number of grains per unit length $n_L$ of a random line drawn across a micrograph. Average grain diameter $d$ is given by 
	$$
	d = \frac{C}{n_LM}
	$$
	Where $M$ is the magnification, and $C$ is some constant greater than 1. Typically 1.5 is adequate.
	\subsection{Noncrystalline Solids: Three-Dimensional Imperfections}
	\subsubsection{Zachariasen Model}
	Illustrates the important features of oxide glass structures. (Recall: Glass refers to a noncrystalline structure that shares characteristics of a ceramic.) \textit{Short Range Order (SRO)} is maintained, in that the the building blocks still retain their shape.  However, the \textit{Long range order (LRO)}, or crystallinity is lost in the glass.  
	
	\subsubsection{Bernal Model}
	Produced by drawing lines between the centers of adjacent atoms.  The resulting polyhedra are irregular in shape and lack any repitition in stacking arrangements. Some evidence for \textit{medium range order (MRO)}, which has effects over the range of a few nanometers, and can introduce structure on a very small scale. 
	
	\section{Diffusion}
	\subsubsection*{Boltzmann's Constant}
	$$
	k = 1.381 \times 10^-23 \frac{m^2 kg}{s^2 K} = 8.617 \times 10 ^{-5} \frac{eV}{K}
	$$
	\subsubsection*{Ideal Gas Constant}
	$$
	R = 8.314 \frac{J}{\text{mol} \cdot K}
	$$
	
	\subsection{Thermally Activated Processes}
	Typically processes increase in rate with temperature.``Rate" = $k$. 
	\begin{large}
		$$\text{rate} = Ce^{\frac{-Q}{RT}}$$
	\end{large}
	Where $C$ is a \textbf{preexponential constant}, independent of temperature.  $Q$ is the \textbf{activation energy}, $R$ is the universal gas constant, and $T$ is the absolute temperature.
	
	\textit{To solve for $Q$:} Take two different temperatures and rates, then divide one by the other.  $C$ cancels, solve for $Q$. 
	
	\subsubsection*{Arrhenius Equation}
	$$
	\ln \text{(rate)} = \ln C - \frac{Q}{R} \frac{1}{T}
	$$
	Making a semi-log plot of $\ln \text{(rate)}$ vs $\frac{1}{T}$ creates a straight line plot of rate data with slope $\frac{Q}{R}$.  $\mathcal Y$ intercept is $\ln C$. 
	
	\subsubsection*{Maxwell-Boltzmann Distribution}
	\begin{large}
		$$
		\mathcal P \propto e^{\frac{\Delta E}{kT}}
		$$
		Where $\mathcal P$ is the probability of finding a molecule at an energy $\Delta E$ greater than the average energy characteristic of a particular temperature, $T$. The energy barrier must be overcome by thermal activation. 
		
		Above equation was developed for gasses, holds true for solids too. Often requires several steps; slowest step will be the rate-limiting step. 
	\end{large}
	\subsection{Thermal Production of Point Defects}
	Point defects occur as a result of thermal vibration of the atoms in a crystal structure. 
	\begin{Large}
		$$
		\frac{n_{\text{defects}}}{n_{\text{sites}}} = Ce^\frac{-E_{\text{defect}}}{kT}
		$$
	\end{Large}
	Where $\frac{n_{\text{defects}}}{n_{\text{sites}}}$ is the ratio of point defects to ideal crystal-lattice sites, C is a preexponential constant, $E_{\text{defect}}$ is the energy needed to create a single point defect, $k$ is Boltzmann's constant, and $T$ is the absolute temperature. 
	
	E depends on the type of defect being considered. 
	
	Concentration of vacancies: 
	\begin{Large}
		$$
		\frac{n_v}{n_{\text{sites}}} = Ce^{\frac{-E_v}{kT}}
		$$
	\end{Large}
	Where $E_v$ is the energy of formation of a single vacancy. 
	\begin{Large}
		$$
		\ln \frac{n_v}{n_{\text{sites}}} = \ln C - \frac{-E_v}{kT}
		$$
	\end{Large}
	\subsection{Point Defects and Solid-State Diffusion}
	Molecules from areas of high concentration to areas of low concentration. \textbf{Vacancy migration} moves particles along vacancies.  Material flows opposite direction from vacancy. Particles can ``Randomly Walk" in any direction with equal probability, but the difference in concentration tends to produce a net flow from areas of high concentration to areas of low concentration. 
	
	\subsubsection*{Fick's First Law}
	$$
	J_x = -D\frac{\delta c}{\delta x}
	$$
	
	Where $J_x$ is the flux, of the diffusing species due to a \textbf{concentration gradient} $\frac{\delta c}{\delta x}$. The proportionality coefficient $D$ is called the \textit{diffusion coefficient} or \textit{diffusivity.} 
	
	\subsubsection*{Fick's Second Law}
	$$
	\frac{\delta c_x}{\delta t} = \frac{\delta }{\delta x} \bigg(D \frac{\delta c_x}{\delta x} \bigg)
	$$
	For most problems it can be assumed that $D$ is independent of $c$, leading to the simplified version:
	$$
	\frac{\delta c_x}{\delta t} = D \frac{\delta^2 c_x}{\delta x^2} 
	$$
	Solutions take the form
	$$
	\frac{c_x - c_0}{c_s - c_0} = 1 - \text{erf} \bigg( \frac{x}{2 \sqrt{Dt}} \bigg)
	$$
	Where $c_0$ is the initial bulk concentration of the diffusing species, and `erf' refers to the \textbf{Gaussian Error Function}. 
	\textit{Gaussian Error Function: }
	$$
	\text{erf}(z) \equiv \frac{2}{\sqrt{\pi}} \int_0^z{e^{-t^2} dt}
	$$
	\textit{Linear Interpolation Formula}
	
	$$\begin{bmatrix}
	x_1 & y_1 \\ 
	x_2 & y_2 \\ 
	x_3 & y_3
	\end{bmatrix}, \mathbf{y_2} = \frac{(x_2-x_1)(y_3-y_1)}{(x_3 - x_1)} + y_1, \: \mathbf{x_2} = \frac{(y_2-y_1)(x_3-x_1)}{(y_3-y_1)} + x_1
	$$
	\subsubsection*{Arrhenius for Diffisivity Data}
	
	\begin{large}
		$$
		D = D_0 e^{\frac{-q}{kT}}
		$$
	\end{large}
	
	\begin{description}
		\item[$D$] is the diffusion coefficient, in $\frac{m^2}{s}$
		\item[$D_0$] is the maximum diffusion coefficient at $\infty$ temperature
		\item[$q$] is the activation energy for defect motion. 
		and $q = E_{\text{defect}} + E_{\text{defect motion}} $
	\end{description} 
	or 
	
	\begin{large}
		$$
		\boxed{D = D_0 e^{\frac{-Q}{RT}}}
		$$
	\end{large}
	\begin{description}
		\item[$Q$] is activation energy per mole of diffusing species
	\end{description}
	\subsection{Steady-State Diffusion}
	\subsubsection*{Concentration Gradient}
	See Fick's $1^{\text{st}}$ law.
	$$
	\frac{\delta c}{\delta x} = \frac{Delta c}{\Delta x} = \frac{c_h - c_l}{0-x_0} = -\frac{c_h-c_l}{x_0}
	$$
	\subsection{Alternate Diffusion Paths}
	Up until now just considered volume diffusion, but diffusion can be much easier along a grain boundary.   Surface diffusion is even easier. 
	
	$$
	Q_{\text{volume}} > Q_{\text{gran boundary}} > Q_{\text{surface}}
	$$
	and
	$$
	D_{\text{volume}} < D_{\text{gran boundary}} < D_{\text{surface}}
	$$
	
	\section{Mechanical Behavior}

	\subsection{Stress Versus Strain}
	\subsubsection*{Metals}
	\emph{Engineering Stress:}
	$$
	\sigma = \frac{P}{A_0}
	$$
	\begin{description}
		\item[$P$] is the load on the original sample.
		\item[$A_0$] is the original cross sectional area
	\end{description}
	
	\emph{Engineering Strain}
	$$
	\epsilon = \frac{l - l_0}{l_0} = \frac{\Delta l}{l_0}
	$$
	\begin{description}
		\item[$l$] is the gage length at a given load
		\item[$l_0$] is the original length. 
	\end{description}
	
	\begin{description}
		\item[Elastic Deformation] is temporary, and is fully recovered when load is removed.
		\item[Plastic Deformation] is permanent, although a small elastic portion may be recovered.
		\item[Plastic Region] is the non-linear portion generated once the total strain exceeds elastic limits.  
		\item[Yield Strength] is defined as the intersection of the deformation curve with a straight line parallel to the elastic portion, and offset $.2\%$ on the strain axis. Yield strength represents the stress necessary to generate $.2\%$ of deformation.
	\end{description}
	\begin{figure}
\centering
		\includegraphics[width=0.7\linewidth]{Images/zkOwknr}
		\label{fig:zkOwknr}
\includegraphics[width=1.1\linewidth]{../../Documents/ShareX/Screenshots/2016-02/chrome_2016-02-21_18-05-32}
 \caption{Yield Strength and Elastic Recovery}
\label{fig:chrome_2016-02-21_18-05-32}
\end{figure}


	\subsubsection*{Young's Modulus, or Modulus of Elasticity}
	The linearity of the stress-strain plot is a graphical statement of the linearity of \textit{Hooke's Law}:
	$$
	\sigma = E \epsilon
	$$
	\begin{description}
		\item[E] is the modulus, which represents the stiffness. 
	\end{description}
	
	\begin{description}
		\item[Specific Strenth] or Strength-to-weight ratio is often preferred in aerospace applications.
		\item[Residual Stress] is defined as the residual stress remaining after loads are removed. 
		\item[Tensile Strength] or \textit{Ultimate Tensile Strength} is the maximum stress. 
		\item[Strain Hardening] is the term for increasing strength with increasing deformation. 
	\end{description}
	
	\subsubsection*{True Stress}
	$$
	\sigma_T = K\epsilon^n_T
	$$
	\begin{description}
		\item[$\sigma_t$] is the true stress
		\item[$\epsilon_T$] Is the true strain
		\item[$K,n$] are constants with values given for a metal or alloy
	\end{description}
	
	\begin{description}
		\item[Ductility] is quantified as the percent elongation at failure: $100 \times \epsilon_{\text{failure}}$. Opposite of ductile is brittle. 
		\item[Toughness] is the combination of strength and ductility. 
		\item[Upper Yield Point] is the transition from elastic region, followed by a non-homogeneous deformation that begins at a point of stress concentration. (Often where the specimen is gripped)
		\item[Lower Yield Point] is the end of the ripple pattern, beginning of general plastic deformation. 
	\end{description}
	
	\subsubsection*{Poisson's Ratio}
	$$
	v = - \frac{\epsilon_x}{\epsilon_z}
	$$
	\begin{description}
		\item[$\epsilon_x,\epsilon_z$] are the strain in the $x$ and $z$ directions. 
	\end{description}
	
	\subsubsection*{Shear Stress:}
	$$
	\tau = \frac{P_s}{A_s}
	$$
	\begin{description}
		\item[$P_s$] is the load on the sample
		\item[$A_s$] is the area of the sample parallel to the applied load. 
	\end{description}
	
	Shear stress produces an angular displacement:
	
	$$
	\gamma = \tan \alpha
	$$
	\begin{description}
		\item[$\gamma$] is the shear strain
		\item[$\alpha$] is the angular displacement.
	\end{description}
	
	
	
	\subsubsection*{Shear Modulus, or Modulus of Rigidity $(G)$}
	$$
	G = \frac{\tau}{\gamma}
	$$
	
	The shear modulus $G$ and the elastic modulus $E$ are related for small strains by Poisson's ratio:
	$$
	E = 2G(1 + v)
	$$ 
	
	\subsubsection*{Ceramics and Glasses}
	\emph{Modulus of Rupture}
	$$
	\textbf{MOR} = \frac{3FL}{2bh^2}
	$$
	\begin{description}
		\item[$F$] Is the applied force
		\item[$b,h,L$] are dimensions defined in the figure below. 
	\end{description}
	\begin{figure}
\centering
\includegraphics[width=0.7\linewidth]{../../Documents/ShareX/Screenshots/2016-02/chrome_2016-02-22_23-27-35}
\end{figure}

\subsubsection*{Griffith Crack Model}
Assumes that there will be numerous elliptical cracks at the surface and/or interior. The highest stress at the tip of such a crank is $\sigma_m$.
$$
\sigma_m \simeq 2\sigma\bigg( \frac{c}{\rho} \bigg)^{{\frac{1}{2}}}
$$
\begin{description}
	\item[$\sigma$] is the applied stress
	\item[$c$] is the crack length
	\item[$\rho$] is the radius of the crack tip
\end{description}

\subsubsection*{Polymers}
	The \textit{flexural strength} is equivalent to the modulus of rupture defined for ceramics.  The \textit{flexural modulus} is:
	$$
	E_{\text{flex}} = \frac{L^3 m}{4bh^3}
	$$
	\begin{description}
		\item[$m$] is the slope of the tangent to the initial straight-line portion of the load deflection curve.  Others defined above.
	\end{description}
	
	\emph{Dynamic Modulus of Elasticity:}
	$$
	E_{\text{dyn}} = CIf^2
	$$
	\begin{description}
		\item[$C$] is a constant defined by specific test geometry
		\item[$I$] is the moment of inertia of the beam and weights
		\item[$f$] is the frequency of vibration for the test
	\end{description}
	
	\subsection{Elastic Deformation}
	Fundamental mechanism at play here is the stretching of atomic bonds. For elastic deformation, compressive and tensile forces often behave similarly.  
	\subsection{Plastic Deformation}  
	Works better with existing deformations and imperfections. The theoretical \textit{critical shear stress} is roughly one order of magnitude less than the bulk \textit{shear modulus, G}. 
	
	Slipping along a plane in the crystal structure is more difficult as the atomic step distances are increased. Increased dislocations \textit{hinder} further dislocations, which is why cold-working a metal can increase the strength.
	 
	Dislocation slips are easier along slip planes with higher atomic density, because the ``road" is smoother.  The number of slip systems dramatically alter the ductility of the material; materials with a greater number of slip systems are far more ductile.  
	
	We can define the ``resolved shear stress", $\tau$ which is the actual stress operating on the slip system. 
	
	\begin{figure}
\centering
\includegraphics[width=0.7\linewidth]{Images/m0rmZXt}
\caption{}
\label{fig:m0rmZXt}
\end{figure}

	$$
	\tau = \sigma \cos( \lambda ) \cos(\phi)
	$$
	Where $\sigma = \frac{F}{A}$ 
	
	A value of $\tau$ great enough to produce slip by dislocation motion is called the \textit{critical resolved shear stress}, and is given by
	
	$$
	\tau_c = \sigma_c \cos (\lambda) \cos (\phi)
	$$
	
	\subsection{Hardness}
	The \textit{hardness test} is a relatively simple alternative to the tensile test from earlier. Often use the ``Brinell Hardness Number" to express the hardness, given by
	
	$$
	BHN = \frac{2P}{\pi D [D - \sqrt{D^2 - d^2}]}
	$$
	\begin{description}
		\item[$P$] is the load
		\item[$D$] is the sphere diameter (mm)
		\item[$d$] is the indent diameter (mm)
	\end{description}
	\subsection{Creep and Stress Relaxation}
	\textit{\textbf{Creep}} can be defined as plastic deformation occurring at high temperature under constant load over a long period. 
	\begin{description}
		\item[Primary Stage] is characterized by a decreasing strain rate.  Relatively rapid increase in length. 
		\item[Secondary Stage] is characterized by straight-line, constant-strain-rate data. 
		\item[Tertiary Stage] strain rate increases due to an increase in true stress.  The comes from a cross-sectional area reduction. 
	\end{description} 
	
	Can view the steady-state creep rate $\dot{\epsilon}$ with activation energy $Q$: 
	
	$$
	\dot{\epsilon} = Ce^{\frac{-Q}{RT}}
	$$
	
	Can also characterize \textit{stress relaxation}, similar to a rubber band.  Use relaxation time $\tau$ defined as the time necessary for the stress $\sigma$ to fall to $\frac{1}{e}$ of the initial stress $\sigma_0$. 
	$$
	\sigma = \sigma_0 e^{\frac{-t}{\tau}}
	$$
	or
	$$
	\frac{1}{\tau} = Ce^{\frac{-Q}{RT}}
	$$
	
	\subsection{Viscoelastic Deformation}
	Viscous behavior of glasses can be defined by the ``viscosity", $\eta$. 
	
	$$
	\frac{F}{A} = \eta \frac{dv}{dx}
	$$
	
	\subsubsection*{Inorganic Glasses}
	$$
	\eta = \eta_0 e^{Q}{RT}
	$$
	
	\subsubsection*{Organic Polymers}
	Modulus of Elasticity is usually plotted instead of viscosity for organic polymers. 
	
\end{document}