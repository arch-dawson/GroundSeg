\documentclass[10pt,letterpaper]{article}
\usepackage[latin1]{inputenc}
\usepackage{amsmath}
\usepackage{amsfonts}
\usepackage{amssymb}
\usepackage{graphicx}
\title{Linear Algebra in Aerospace Engineering}
\author{Dawson Beatty}
\begin{document}
	\maketitle
	\section*{Rotations}
	One of the main applications of linear algebra within Aerospace engineering is the use of rotation matrices.  Rotation matrices can be used to describe the roll, pitch, and yaw of a spacecraft, which represent rotation about the $x$, $y$, and $z$ axes respectively. This from Glenn Murray's \textit{Rotation About an Arbitrary Axis in 3 Dimensions}, \cite{Rotations}\\
	
	\textbf{Roll:}
	$$
	R_x(\gamma) = \begin{bmatrix}
	1 & 0 & 0 \\ 0 & \cos \gamma & -\sin \gamma \\ 0 & \sin \gamma & \cos \gamma 
	\end{bmatrix}
	$$
	
	\textbf{Pitch:}
	$$
	R_y( \beta ) = \begin{bmatrix}
	\cos \beta & 0 & \sin \beta \\ 0 & 1 & 0 \\ - \sin \beta & 0 & \cos \beta
	\end{bmatrix}
	$$
	
	\textbf{Yaw:}
	$$
	R_z( \alpha) = \begin{bmatrix}
	\cos \alpha & - \sin \alpha & 0 \\ \sin \alpha & \cos \alpha & 0 \\ 0 & 0 & 1
	\end{bmatrix}
	$$
	
	\section*{Statics}
	
	Statics also makes use of the principles of linear algebra.  The whole idea of statics is taking the sum of the forces in the $x$ and $y$ directions-- as well as the net torque-- to be zero. 
	$$
	\sum F_x = 0 \qquad \sum F_y = 0 \qquad \tau_{\text{net}} = 0
	$$ 
	
	These formulas provided by Professor Wolfson in his \textit{Essential university physics} \cite{Statics}.  This usually involves identifying all the forces in the problem, and summing them up to zero.  This is at it's heart a giant homogeneous linear algebra problem. It would also be \textit{really} interesting to see if the solution to an engineering problem you found was guaranteed to be unique, or if there might be another solution that you're missing. 
	
	\section*{Component Connections}
	
	Again this is not strictly an Aerospace engineering application, but more of an engineering application in general. When working with a large system, it's helpful to know which parts connect ot each other, and the distances between each of the components. 
	
	Problems like this can be solved by creating a matrix with one row and one column for each component, then putting a $1$ for connections between components, and a $0$ for components that don't connect. \cite{Graph}

	
	\begin{thebibliography}{9}
		\bibitem{Rotations}
		Murray, G. (2013, June 6). \textit{Rotation About an Arbitrary Axis in 3 Dimensions}. Retrieved March 10, 2016, from http://inside.mines.edu/fshome/gmurray/ArbitraryAxisRotation/
		\bibitem{Statics}
		Wolfson, R. (2012). Essential university physics. Boston: Addison-Wesley.
		\bibitem{Graph}
		Trudeau, R. J. (1993). Introduction to graph theory. New York: Dover Pub.
	\end{thebibliography}
\end{document}
